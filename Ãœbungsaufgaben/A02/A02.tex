\documentclass{article}

\usepackage{fancyhdr}
\usepackage{extramarks}
\renewcommand{\thispagestyle}[1]{} % damit Titelseite Kopfzeile bekommt
\usepackage{amsmath}
\usepackage{amsthm}
\usepackage{amsfonts}
\usepackage{tikz}
\usepackage[plain]{algorithm}
\usepackage{algpseudocode}
\usepackage{listings}
\lstset{language=C++, numbers=left, frame=single,}
\usepackage[utf8]{inputenc}
\usepackage{paralist}

\usetikzlibrary{automata,positioning}

%
% Basic Document Settings
%

\topmargin=-0.45in
\evensidemargin=0in
\oddsidemargin=0in
\textwidth=6.5in
\textheight=9.0in
\headsep=0.25in

\linespread{1.1}

\pagestyle{fancy}
\lhead{\hmwkGroup\\\hmwkTitle}
\rhead{\hmwkAuthorName}
\lfoot{\lastxmark}
\cfoot{\thepage}

\renewcommand\headrulewidth{0.4pt}
\renewcommand\footrulewidth{0.4pt}

\setlength\parindent{0pt}

%
% Create Problem Sections
%

\newcommand{\enterProblemHeader}[1]{
    \nobreak\extramarks{}{Aufgabe \arabic{#1} wird auf nächster Seite fortgesetzt\ldots}\nobreak{}
    \nobreak\extramarks{Aufgabe \arabic{#1} (fortgesetzt)}{Aufgabe \arabic{#1} wird auf nächster Seite fortgesetzt\ldots}\nobreak{}
}

\newcommand{\exitProblemHeader}[1]{
    \nobreak\extramarks{Aufgabe \arabic{#1} (fortgesetzt)}{Aufgabe \arabic{#1} wird auf nächster Seite fortgesetzt\ldots}\nobreak{}
    \stepcounter{#1}
    \nobreak\extramarks{Aufgabe \arabic{#1}}{}\nobreak{}
}

\setcounter{secnumdepth}{0}
\newcounter{partCounter}
\newcounter{homeworkProblemCounter}
\setcounter{homeworkProblemCounter}{1}
\nobreak\extramarks{Aufgabe \arabic{homeworkProblemCounter}}{}\nobreak{}

%
% Homework Problem Environment
%
% This environment takes an optional argument. When given, it will adjust the
% problem counter. This is useful for when the problems given for your
% assignment aren't sequential. See the last 3 problems of this template for an
% example.
%

\newenvironment{homeworkProblem}[1][-1]{
    \ifnum#1>0
        \setcounter{homeworkProblemCounter}{#1}
    \fi
    \section{Aufgabe \arabic{homeworkProblemCounter}}
    \setcounter{partCounter}{1}
    \enterProblemHeader{homeworkProblemCounter}
}{
    \exitProblemHeader{homeworkProblemCounter}
}

%
% Homework Details
%   - Title
%   - Due date
%   - Class
%   - Section/Time
%   - Instructor
%   - Author
%

\date{22.04.2018}
\newcommand{\hmwkTitle}{Übungsblatt\ \#2}
\newcommand{\hmwkGroup}{Übungsgruppe 16}
\newcommand{\hmwkDueDate}{19. April 2018}
\newcommand{\hmwkClass}{Datenstrukturen und Algorithmen}
\newcommand{\hmwkAuthorName}{\textbf{Finn Hess (378104), Jan Knichel (??????), Paul Orschau (381085)}}

%
% Title Page
%

\title{
    \vspace{2in}
    \textmd{\textbf{\hmwkClass:\ \hmwkTitle}}\\
    \normalsize\vspace{0.1in}\small{Abgabe\ am\ \hmwkDueDate\ }\\
    \vspace{3in}
}

\author{\hmwkAuthorName}

\renewcommand{\part}[1]{\textbf{\large Part \Alph{partCounter}}\stepcounter{partCounter}\\}

%
% Various Helper Commands
%

% Useful for algorithms
\newcommand{\alg}[1]{\textsc{\bfseries \footnotesize #1}}

% For derivatives
\newcommand{\deriv}[1]{\frac{\mathrm{d}}{\mathrm{d}x} (#1)}

% For partial derivatives
\newcommand{\pderiv}[2]{\frac{\partial}{\partial #1} (#2)}

% Integral dx
\newcommand{\dx}{\mathrm{d}x}

% Alias for the Solution section header
\newcommand{\loesung}{\textbf{\large Lösung}}



%
% Start of Document
%

\begin{document}

  \maketitle

  \pagebreak

  \begin{homeworkProblem}
      
      Wir definieren die Quasiordnung $\sqsubseteq$ auf Funktionen als
      
      $ f \sqsubseteq g \Leftrightarrow f \in \mathcal O(g)$.
      
      \begin{enumerate}[a)]
          \item Beweisen Sie, dass $\sqsubseteq$ eine Quasiordnung ist, das heißt dass $\sqsubseteq$ reflexiv und transitiv ist.
          \item Sortieren Sie einige Funktionen in aufsteigender Reihenfolge bezüglich der Quasiordnung $\sqsubseteq$.
      \end{enumerate}

      \loesung
      
      \textbf{Teil a)}
      
      \textbf{Teil b)}
      
  \end{homeworkProblem}

  \pagebreak

  \begin{homeworkProblem}
      Beweisen oder wiederlegen Sie folgende Aussagen:
      
      \begin{enumerate}[a)]
        \item $\frac{1}{4}n^3-7n+17 \in \mathcal O(n^3)$
        \item $n^4 \in \mathcal O(2n^4+3n^2+42)$
        \item $\log(n) \in \mathcal O(n)$
        \item $\forall \epsilon>0: \log(n) \in \mathcal O(n^\epsilon)$
        \item $a^n \in \Theta(b^n)$, für zwei beliebige Konstanten $a,b>1$
      \end{enumerate}
      
      \loesung
      
      \textbf{Teil a)}
                      
      \textbf{Teil b)}
                      
      \textbf{Teil c)}
                      
      \textbf{Teil d)}
                      
      \textbf{Teil e)}
                      
  \end{homeworkProblem}

  \pagebreak

  \begin{homeworkProblem}

    \begin{enumerate}[a)]
      \item Zeigen oder wiederlegen Sie:
      \begin{center}
        $o(g(n)) \cap \Theta(g(n)) = \emptyset$
      \end{center}
      \item Zeigen oder widerlegen Sie:
      \begin{center}
        $f(n) \in \Omega(g(n)) \land f(n) \in \mathcal O(h(n)) \implies g \in \Theta(h(n))$
      \end{center}    
    \end{enumerate}

    \loesung
    
    \textbf{Teil a)}

    \textbf{Teil b)}

  \end{homeworkProblem}

  \pagebreak

  \begin{homeworkProblem}
    
    Gegeben sei ein Algorithmus, der für ein Array von Booleans überprüft, ob alle Einträge wahr sind:
    
    \lstinputlisting{Code/4_Example.cpp}
    
    Bei Betrachtung der Laufzeit wird angenommen, dass Vergleiche (z.B. $x<y$ oder $b==0$) jeweils eine Zeiteinheit benötigen. Die Laufzeit aller anderen Operationen wird vernachlässigt.
    
    Sei $n$ die Länge des Arrays $E$.
    
    \begin{enumerate}[a)]
      \item Bestimmen Sie in Abhängigkeit von $n$ die Best-case Laufzeit $B(n)$.
      \item Bestimmen Sie in Abhängigkeit von $n$ die Worst-case Laufzeit $W(n)$.
      \item Bestimmen Sie in Abhängigkeit von $n$ die Average-case Laufzeit $A(n)$. Hierzu nehmen wir eine uniforme Verteilung der möglichen Eingaben $D_n$ an, d.h. jede beliebige Eingabe $E \in D_n$ tritt mit Wahrscheinlichkeit $1/|D_n|$ auf.
      \item Geben Sie einen äquivalenten Algorithmus an, dessen Average-case Laufzeit ab einer gewissen Eingabelänge kleiner ist. Hierzu nehmen wir wieder eine uniforme Verteilung der möglichen Eingaben an. Begründen Sie Ihre Antwort kurz. Sie müssen nicht die Average-case Laufzeit ihres Algorithmus berechnen.
      \item Gibt es einen äquivaltenen Algorithmus, dessen Worst-case Laufzeit in $o(n)$ liegt? Begründen Sie ihre Antwort.
    \end{enumerate}
    
    \loesung
    
    \textbf{Teil a)}
    
    \textbf{Teil b)}
    
    \textbf{Teil c)}
    
    \textbf{Teil d)}
    
    \textbf{Teil e)}
    
  \end{homeworkProblem}

\end{document}