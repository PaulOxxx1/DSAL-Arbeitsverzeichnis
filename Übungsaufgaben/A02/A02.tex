\documentclass{article}

\usepackage{fancyhdr}
\usepackage{extramarks}
\renewcommand{\thispagestyle}[1]{} % damit Titelseite Kopfzeile bekommt
\usepackage{amsmath}
\usepackage{amsthm}
\usepackage{amsfonts}
\usepackage{tikz}
\usepackage[plain]{algorithm}
\usepackage{algpseudocode}
\usepackage{listings}
\lstset{language=C++, numbers=left, frame=single,}
\usepackage[utf8]{inputenc}
\usepackage{paralist}

\usetikzlibrary{automata,positioning}

%
% Basic Document Settings
%

\topmargin=-0.45in
\evensidemargin=0in
\oddsidemargin=0in
\textwidth=6.5in
\textheight=9.0in
\headsep=0.25in

\linespread{1.1}

\pagestyle{fancy}
\lhead{\hmwkGroup\\\hmwkTitle}
\rhead{\hmwkAuthorName}
\lfoot{\lastxmark}
\cfoot{\thepage}

\renewcommand\headrulewidth{0.4pt}
\renewcommand\footrulewidth{0.4pt}

\setlength\parindent{0pt}

%
% Create Problem Sections
%

\newcommand{\enterProblemHeader}[1]{
    \nobreak\extramarks{}{Aufgabe \arabic{#1} wird auf nächster Seite fortgesetzt\ldots}\nobreak{}
    \nobreak\extramarks{Aufgabe \arabic{#1} (fortgesetzt)}{Aufgabe \arabic{#1} wird auf nächster Seite fortgesetzt\ldots}\nobreak{}
}

\newcommand{\exitProblemHeader}[1]{
    \nobreak\extramarks{Aufgabe \arabic{#1} (fortgesetzt)}{Aufgabe \arabic{#1} wird auf nächster Seite fortgesetzt\ldots}\nobreak{}
    \stepcounter{#1}
    \nobreak\extramarks{Aufgabe \arabic{#1}}{}\nobreak{}
}

\setcounter{secnumdepth}{0}
\newcounter{partCounter}
\newcounter{homeworkProblemCounter}
\setcounter{homeworkProblemCounter}{1}
\nobreak\extramarks{Aufgabe \arabic{homeworkProblemCounter}}{}\nobreak{}

%
% Homework Problem Environment
%
% This environment takes an optional argument. When given, it will adjust the
% problem counter. This is useful for when the problems given for your
% assignment aren't sequential. See the last 3 problems of this template for an
% example.
%

\newenvironment{homeworkProblem}[1][-1]{
    \ifnum#1>0
        \setcounter{homeworkProblemCounter}{#1}
    \fi
    \section{Aufgabe \arabic{homeworkProblemCounter}}
    \setcounter{partCounter}{1}
    \enterProblemHeader{homeworkProblemCounter}
}{
    \exitProblemHeader{homeworkProblemCounter}
}

%
% Homework Details
%   - Title
%   - Due date
%   - Class
%   - Section/Time
%   - Instructor
%   - Author
%

\date{22.04.2018}
\newcommand{\hmwkTitle}{Übungsblatt\ \#2}
\newcommand{\hmwkGroup}{Übungsgruppe 16}
\newcommand{\hmwkDueDate}{26. April 2018}
\newcommand{\hmwkClass}{Datenstrukturen und Algorithmen}
\newcommand{\hmwkAuthorName}{\textbf{Finn Hess (378104), Jan Knichel (377779), Paul Orschau (381085)}}

%
% Title Page
%

\title{
    \vspace{2in}
    \textmd{\textbf{\hmwkClass:\ \hmwkTitle}}\\
    \normalsize\vspace{0.1in}\small{Abgabe\ am\ \hmwkDueDate\ }\\
    \vspace{3in}
}

\author{\hmwkAuthorName}

\renewcommand{\part}[1]{\textbf{\large Part \Alph{partCounter}}\stepcounter{partCounter}\\}

%
% Various Helper Commands
%

% Useful for algorithms
\newcommand{\alg}[1]{\textsc{\bfseries \footnotesize #1}}

% For derivatives
\newcommand{\deriv}[1]{\frac{\mathrm{d}}{\mathrm{d}x} (#1)}

% For partial derivatives
\newcommand{\pderiv}[2]{\frac{\partial}{\partial #1} (#2)}

% Integral dx
\newcommand{\dx}{\mathrm{d}x}

% Alias for the Solution section header
\newcommand{\loesung}{\textbf{\large Lösung}}



%
% Start of Document
%

\begin{document}

  \maketitle

  \pagebreak

  \begin{homeworkProblem}
      
      Wir definieren die Quasiordnung $\sqsubseteq$ auf Funktionen als
      
      $ f \sqsubseteq g \Leftrightarrow f \in \mathcal O(g)$.
      
      \begin{enumerate}[a)]
          \item Beweisen Sie, dass $\sqsubseteq$ eine Quasiordnung ist, das heißt dass $\sqsubseteq$ reflexiv und transitiv ist.
          \item Sortieren Sie einige Funktionen in aufsteigender Reihenfolge bezüglich der Quasiordnung $\sqsubseteq$.
      \end{enumerate}

      \loesung
      
      \textbf{Teil a)}
      
      $\sqsubseteq$ ist reflexiv, denn $\forall f$ gilt:
      
      $f \sqsubseteq f
      \Leftrightarrow f \in \mathcal O(f)
      \Leftrightarrow \limsup\limits_{n\rightarrow\infty} \frac{f(n)}{f(n)} = 1 < \infty $ \qed
      
      $\sqsubseteq$ ist transitiv, denn $\forall f,g,h$ mit $f \sqsubseteq g$ und $g \sqsubseteq h$ gilt:
      
      $f \sqsubseteq h
      \Leftrightarrow \limsup\limits_{n\rightarrow\infty}  \frac{f(n)}{h(n)}
      = \limsup\limits_{n\rightarrow\infty} [\frac{f(n)}{g(n)}*\frac{g(n)}{h(n)}]
      \leq \limsup\limits_{n\rightarrow\infty} \frac{f(n)}{g(n)} * \limsup\limits_{n\rightarrow\infty} \frac{g(n)}{h(n)}
      \stackrel{{f \sqsubseteq g}\atop{g \sqsubseteq h}}{=} c_1 * c_2 < \infty $ \qed
      
      \textbf{Teil b)}
      
      $ 0
      \sqsubseteq 4
      \sqsubseteq 2^{9000}
      \sqsubseteq \log{n}
      \stackrel{\text{1)}}{\sqsubseteq} n\log{n}
      \stackrel{\text{2)}}{\sqsubseteq} n\sqrt{n}
      \stackrel{\text{3)}}{\sqsubseteq} n^{2}
      \stackrel{\text{4)}}{\sqsubseteq} \sum\limits_{i=0}^{n} \frac{14i^2}{1+i}
      \sqsubseteq n^{2}\log{n}
      \sqsubseteq n^{3} 
      \sqsubseteq \frac{n^3}{2} 
      \sqsubseteq 2^{n} 
      \sqsubseteq n! 
      \sqsubseteq n^{n} $
      
      \begin{enumerate}
      \item $ \limsup\limits_{n\rightarrow\infty} \frac{\log{n}}{n\log{n}}
            \stackrel{\text{l'Hopital}}{=} \limsup\limits_{n\rightarrow\infty} \frac{\frac{1}{n}}{n*\frac{1}{n}+1*\log{n}}
            = \limsup\limits_{n\rightarrow\infty} \frac{1}{n+n*\log{n}} = 0 < \infty $ \qed
      
      \item $ \limsup\limits_{n\rightarrow\infty} \frac{n\log{n}}{n\sqrt{n}}
            = \limsup\limits_{n\rightarrow\infty} \frac{ \log{n}}{ \sqrt{n}}
            \stackrel{\text{l'Hopital}}{=} \limsup\limits_{n\rightarrow\infty} \frac{\frac{1}{n}}{\frac{1}{2\sqrt{n}}}
            = \limsup\limits_{n\rightarrow\infty} \frac{2\sqrt{n}}{n}
            = \limsup\limits_{n\rightarrow\infty} \frac{2}{\sqrt{n}} = 0 < \infty $ \qed 

      \item $ \limsup\limits_{n\rightarrow\infty} \frac{n\sqrt{n}}{n^2}
            = \limsup\limits_{n\rightarrow\infty} \frac{\sqrt{n}}{n}
            = \limsup\limits_{n\rightarrow\infty} \frac{1}{\sqrt{n}} = 0 < \infty $ \qed 
             
      \item $ \limsup\limits_{n\rightarrow\infty} \frac{n^2}{\sum\limits_{i=0}^{n} \frac{14i^2}{1+i}}
            = \limsup\limits_{n\rightarrow\infty} \frac{n^2}{14\sum\limits_{i=0}^{n} [i-1+\frac{1}{i+1}] }
            = \frac{1}{14} * \limsup\limits_{n\rightarrow\infty} \frac{n^2}{\frac{n(n+1)}{2} -n +\sum\limits_{i=0}^{n} \frac{1}{i+1} }
            = \frac{1}{14} * \limsup\limits_{n\rightarrow\infty} \frac{n^2}{\frac{n^2-n}{2} +\sum\limits_{i=0}^{n} \frac{1}{i+1} }
            \\ \leq \frac{1}{7} * \limsup\limits_{n\rightarrow\infty} \frac{n^2}{n^2-n}
            = \frac{1}{7} * \limsup\limits_{n\rightarrow\infty} \frac{1}{1-\frac{1}{n}} = \frac{1}{7} < \infty $ \qed 
                     
      \end{enumerate}
      
  \end{homeworkProblem}

  \pagebreak

  \begin{homeworkProblem}
      Beweisen oder wiederlegen Sie folgende Aussagen:
      
      \begin{enumerate}[a)]
        \item $\frac{1}{4}n^3-7n+17 \in \mathcal O(n^3)$
        \item $n^4 \in \mathcal O(2n^4+3n^2+42)$
        \item $\log(n) \in \mathcal O(n)$
        \item $\forall \epsilon>0: \log(n) \in \mathcal O(n^\epsilon)$
        \item $a^n \in \Theta(b^n)$ für zwei beliebige Konstanten $a,b>1$
      \end{enumerate}
      
      \loesung
      
      \textbf{Teil a)}
      
      Aussage wahr, da
       
      $ \limsup\limits_{n\rightarrow\infty} \frac{\frac{1}{4}n^3-7n+17}{n^3}
      = \limsup\limits_{n\rightarrow\infty} [\frac{1}{4}-\frac{7}{n^2}+\frac{17}{n^3}]
      = \frac{1}{4} < \infty $ \qed
                      
      \textbf{Teil b)}
      
      Aussage wahr, da
       
      $ \limsup\limits_{n\rightarrow\infty} \frac{n^4}{2n^4+3n^2+42}
      = \limsup\limits_{n\rightarrow\infty} \frac{1}{2+\frac{3}{n^2}+\frac{42}{n^4}}
      = \frac{1}{2} < \infty $ \qed
                      
      \textbf{Teil c)}
      
      Aussage wahr, da
       
      $ \limsup\limits_{n\rightarrow\infty} \frac{\log{n}}{n}
      \stackrel{\text{l'Hopital}}{=} \limsup\limits_{n\rightarrow\infty} \frac{\frac{1}{n}}{1}
      = \limsup\limits_{n\rightarrow\infty} \frac{1}{n} = 0 < \infty $ \qed
                      
      \textbf{Teil d)}
      
      Aussage wahr, da
       
      $ \limsup\limits_{n\rightarrow\infty} \frac{\log{n}}{n^{\epsilon}}
      \stackrel{\text{l'Hopital}}{=} \limsup\limits_{n\rightarrow\infty} \frac{\frac{1}{n}}{\epsilon n^{\epsilon-1}}
      = \limsup\limits_{n\rightarrow\infty} \frac{1}{\epsilon n^{\epsilon}}
      \stackrel{(*)}{=} 0 < \infty $ \qed
      
      $(*)$ Wir zeigen Konvergenz gegen 0:
      
      $ \forall \epsilon>0 : \exists n_0(\epsilon) \in \mathbb N : |\frac{1}{an^a}| < \epsilon
      \Leftrightarrow ... \Leftrightarrow \sqrt[a]{\frac{1}{a\epsilon}} < n $      
      
      \textbf{Teil e)}
      
      Aussage falsch, seien z.B. $a=e^3$ und $b=e^2$, dann gilt
           
      $ \lim\limits_{n\rightarrow\infty} \frac{a^n}{b^n}
      = \lim\limits_{n\rightarrow\infty} \frac{e^{3n}}{e^{2n}}
      = \lim\limits_{n\rightarrow\infty} e^{3n-2n}
      = \lim\limits_{n\rightarrow\infty} e^n = \infty $ \qed
                      
  \end{homeworkProblem}

  \pagebreak

  \begin{homeworkProblem}

    \begin{enumerate}[a)]
      \item Zeigen oder wiederlegen Sie:
      \begin{center}
        $o(g(n)) \cap \Theta(g(n)) = \emptyset$
      \end{center}
      \item Zeigen oder widerlegen Sie:
      \begin{center}
        $f(n) \in \Omega(g(n)) \land f(n) \in \mathcal O(h(n)) \implies g \in \Theta(h(n))$
      \end{center}    
    \end{enumerate}

    \loesung
    
    \textbf{Teil a)}
    
    Sei $ f(n)\in o(g(n)) $ beliebig. Das heißt es gilt $ \forall c>0, \exists n_0 $ mit $ \forall n\geq n_0 : 0 \leq f(n) < c*g(n) $.
    
    Nun zeigen wir, dass $f(n)$ nicht in $\Theta (g(n))$ liegen kann.
    
    $f(n) \in \Theta (g(n)) \Leftrightarrow \exists c_1,c_2>0, n_0 $ mit $ \forall n \geq n_0 : c_1*g(n) \leq f(n) \leq c_2*g(n) $.
    
    Da $f\in o(g(n))$ gibt es auch zu jedem $c_1$ ein $n_0$, so dass $f(n) < c_1*g(n)$.
    
    Daher kann kein $c_1$ existieren, so dass $c_1*g(n) \leq f(n)$ ab einem Index $n_0$ gilt. Somit kann dieses $f$ nicht in $\Theta (g(n))$ liegen. Da $f$ beliebig war, ist die Aussage ist gezeigt. \qed

    \textbf{Teil b)}
    
    Gegenbeispiel: Sei $f(n) = n$, $g(n) = 1$ und $h(n) = n^2$. Dann gilt zwar $f(n) \in \Omega(g(n)) \land f(n) \in \mathcal O(h(n)) $, allerdings nicht $g \in \Theta(h(n))$. \qed

  \end{homeworkProblem}

  \pagebreak

  \begin{homeworkProblem}
    
    Gegeben sei ein Algorithmus, der für ein Array von Booleans überprüft, ob alle Einträge wahr sind:
    
    \lstinputlisting{Code/4_Example.cpp}
    
    Bei Betrachtung der Laufzeit wird angenommen, dass Vergleiche (z.B. $x<y$ oder $b==0$) jeweils eine Zeiteinheit benötigen. Die Laufzeit aller anderen Operationen wird vernachlässigt.
    
    Sei $n$ die Länge des Arrays $E$.
    
    \begin{enumerate}[a)]
      \item Bestimmen Sie in Abhängigkeit von $n$ die Best-case Laufzeit $B(n)$.
      \item Bestimmen Sie in Abhängigkeit von $n$ die Worst-case Laufzeit $W(n)$.
      \item Bestimmen Sie in Abhängigkeit von $n$ die Average-case Laufzeit $A(n)$. Hierzu nehmen wir eine uniforme Verteilung der möglichen Eingaben $D_n$ an, d.h. jede beliebige Eingabe $E \in D_n$ tritt mit Wahrscheinlichkeit $1/|D_n|$ auf.
      \item Geben Sie einen äquivalenten Algorithmus an, dessen Average-case Laufzeit ab einer gewissen Eingabelänge kleiner ist. Hierzu nehmen wir wieder eine uniforme Verteilung der möglichen Eingaben an. Begründen Sie Ihre Antwort kurz. Sie müssen nicht die Average-case Laufzeit ihres Algorithmus berechnen.
      \item Gibt es einen äquivaltenen Algorithmus, dessen Worst-case Laufzeit in $o(n)$ liegt? Begründen Sie ihre Antwort.
    \end{enumerate}
    
    \loesung
    
    Zeilen im Code, welche Zeiteinheiten benötigen: \textbf{2, 7, 8, 10}.
    
    Falls das Array \textbf{leer} ist ($n=0$), hat der Algorithmus immer eine Laufzeit von 1 (nur Zeile 2).
    
    Das heißt $B(0)=W(0)=A(0)=1$.
    
    Falls das Array \textbf{höchstens n-1} "true"-Werte enthält, ist die Laufzeit stets
    
    $X(n,a) = \underbrace{1}_2
            + \underbrace{(n+1)}_7
            + \underbrace{n}_8
            + \underbrace{a}_{10}
            = 2n + 2 + a $
          
    wobei $a\in[0,n]$ der Anzahl von "true"-Werten im Array entspricht. Das liegt daran, dass der Algorithmus den ganzen Array abläuft ... statt bei dem ersten gefundenen "false"-Wert abzubrechen. Für jeden gefundenen "true"-Wert kommt ein weiterer Vergleich in Zeile 10 hinzu.
    
    Falls das Array \textbf{genau n} "true"-Werte enthält, wird Zeile 7 einmal weniger ausgeführt, da der Algorithmus bereits in Zeile 11 terminiert. Die Laufzeit entspricht dann 
    
    $X(n,n) = \underbrace{1}_2
            + \underbrace{n}_7
            + \underbrace{n}_8
            + \underbrace{n}_{10}
            = 3n + 1 $
    
    \textbf{Teil a)}
    
    Best-case Laufzeit: $B(n) = X(n,0) = 2n + 2$
    
    \textbf{Teil b)}
    
    Worst-case Laufzeit: $W(n) = X(n,n) = X(n,n-1) = 3n + 1$
    
    \textbf{Teil c)}
    
    Average-case Laufzeit: $A(n)
    = \sum\limits_{I \in D_n} Pr(I)*t(I)
    = \sum\limits_{I \in D_n} \frac{1}{|D_n|}*t(I)
    = \frac{1}{2^n}\sum\limits_{I \in D_n}t(I)
    = \frac{1}{2^n}\sum\limits_{i=0}^n {n \choose i}X(n,i)
    = \frac{1}{2^n}[\sum\limits_{i=0}^{n-1} {n \choose i}X(n,i) + X(n,n)]
    = \frac{1}{2^n}[\sum\limits_{i=0}^{n-1} {n \choose i}(2n+2+i) + (3n+1)] $

    \textbf{Teil d)}
    
    \lstinputlisting{Code/4_Improved.cpp}

    \textbf{Teil e)}
    
    Einen äquivalenten Algorithmus, dessen Worst-case Laufzeit in $o(n)$ liegt, kann es nicht geben, denn im schlimmsten Fall muss der Algorithmus alle Werte des Arrays mindestens einmal überprüfen.
    Der Algorithmus kann keine "Annahmen" über einzelne Werte machen, bzw. man kann nicht auf den Wert eines Eintrag des Arrays schließen, ohne ihn wirklich betrachtet zu haben.
    Da das Array $n$ Werte hat, muss die Worst-case Laufzeit des besten Algorithmus in $\mathcal O(n)$ liegen!
    
  \end{homeworkProblem}

\end{document}