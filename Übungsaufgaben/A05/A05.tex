\documentclass{article}

\usepackage{fancyhdr}
\usepackage{extramarks}
\renewcommand{\thispagestyle}[1]{} % damit Titelseite Kopfzeile bekommt
\usepackage{amsmath}
\usepackage{amsthm}
\usepackage{amsfonts}
\usepackage{tikz}
\usepackage[plain]{algorithm}
\usepackage{algpseudocode}
\usepackage{listings}
\lstset{language=C++, numbers=left, frame=single,}
\usepackage[utf8]{inputenc}
\usepackage{paralist}
\usepackage{tikz}
\usepackage{multicol}

\usetikzlibrary{automata,positioning}

%
% Basic Document Settings
%

\topmargin=-0.45in
\evensidemargin=0in
\oddsidemargin=0in
\textwidth=6.5in
\textheight=9.0in
\headsep=0.25in

\linespread{1.1}

\pagestyle{fancy}
\lhead{\hmwkGroup\\\hmwkTitle}
\rhead{\hmwkAuthorName}
\lfoot{\lastxmark}
\cfoot{\thepage}

\renewcommand\headrulewidth{0.4pt}
\renewcommand\footrulewidth{0.4pt}

\setlength\parindent{0pt}

%
% Create Problem Sections
%

\newcommand{\enterProblemHeader}[1]{
    \nobreak\extramarks{}{Aufgabe \arabic{#1} wird auf nächster Seite fortgesetzt\ldots}\nobreak{}
    \nobreak\extramarks{Aufgabe \arabic{#1} (fortgesetzt)}{Aufgabe \arabic{#1} wird auf nächster Seite fortgesetzt\ldots}\nobreak{}
}

\newcommand{\exitProblemHeader}[1]{
    \nobreak\extramarks{Aufgabe \arabic{#1} (fortgesetzt)}{Aufgabe \arabic{#1} wird auf nächster Seite fortgesetzt\ldots}\nobreak{}
    \stepcounter{#1}
    \nobreak\extramarks{Aufgabe \arabic{#1}}{}\nobreak{}
}

\setcounter{secnumdepth}{0}
\newcounter{partCounter}
\newcounter{homeworkProblemCounter}
\setcounter{homeworkProblemCounter}{1}
\nobreak\extramarks{Aufgabe \arabic{homeworkProblemCounter}}{}\nobreak{}

%
% Homework Problem Environment
%
% This environment takes an optional argument. When given, it will adjust the
% problem counter. This is useful for when the problems given for your
% assignment aren't sequential. See the last 3 problems of this template for an
% example.
%

\newenvironment{homeworkProblem}[1][-1]{
    \ifnum#1>0
        \setcounter{homeworkProblemCounter}{#1}
    \fi
    \section{Aufgabe \arabic{homeworkProblemCounter}}
    \setcounter{partCounter}{1}
    \enterProblemHeader{homeworkProblemCounter}
}{
    \exitProblemHeader{homeworkProblemCounter}
}

%
% Homework Details
%   - Title
%   - Due date
%   - Class
%   - Section/Time
%   - Instructor
%   - Author
%

\date{13.5.2018}
\newcommand{\hmwkTitle}{Übungsblatt\ \#5}
\newcommand{\hmwkGroup}{Übungsgruppe 16}
\newcommand{\hmwkDueDate}{17. Mai 2018}
\newcommand{\hmwkClass}{Datenstrukturen und Algorithmen}
\newcommand{\hmwkAuthorName}{\textbf{Finn Hess (378104), Jan Knichel (377779), Paul Orschau (381085)}}

%
% Title Page
%

\title{
    \vspace{2in}
    \textmd{\textbf{\hmwkClass:\ \hmwkTitle}}\\
    \normalsize\vspace{0.1in}\small{Abgabe\ am\ \hmwkDueDate\ }\\
    \vspace{3in}
}

\author{\hmwkAuthorName}

\renewcommand{\part}[1]{\textbf{\large Part \Alph{partCounter}}\stepcounter{partCounter}\\}

%
% Various Helper Commands
%

% Useful for algorithms
\newcommand{\alg}[1]{\textsc{\bfseries \footnotesize #1}}

% For derivatives
\newcommand{\deriv}[1]{\frac{\mathrm{d}}{\mathrm{d}x} (#1)}

% For partial derivatives
\newcommand{\pderiv}[2]{\frac{\partial}{\partial #1} (#2)}

% Integral dx
\newcommand{\dx}{\mathrm{d}x}

% Alias for the Solution section header
\newcommand{\loesung}{\textbf{\large Lösung}}

% Parts
\newcommand{\teil}[1]{\vspace{15pt}\textbf{Teil #1}}


%
% Start of Document
%

\begin{document}

  \maketitle

  \pagebreak

  \begin{homeworkProblem}
    


  \end{homeworkProblem}
  
  \pagebreak

  \begin{homeworkProblem}
    


  \end{homeworkProblem}
  
  \pagebreak

  \begin{homeworkProblem}
    


    
  \end{homeworkProblem}
    
  \pagebreak

  \begin{homeworkProblem}
    
    \teil{a)}
    
    \begin{itemize}
      \item \textbf{9} / 3 / 5 / 1 / 10 / 4 / 6
      \item \textbf{3 / 9} / 5 / 1 / 10 / 4 / 6
      \item \textbf{3 / 5 / 9} / 1 / 10 / 4 / 6
      \item \textbf{1 / 3 / 5 / 9} / 10 / 4 / 6
      \item \textbf{1 / 3 / 5 / 9 / 10} / 4 / 6
      \item \textbf{1 / 3 / 4 / 5 / 9 / 10} / 6
      \item \textbf{1 / 3 / 4 / 5 / 6 / 9 / 10}
    \end{itemize}

    \teil{b)}
    
    Lösung des Dutch Flag Problem:

    \lstinputlisting{Code/4_Dutch.cpp}
    
    Der Algorithmus kann ein Array in 3 Kategorien sortieren.
    Die Idee bei 5 verschiedenen Farben besteht darin, zunächst in die Kategorien "Blau", "Rot" und "Alles Andere" zu sortieren, und dann den Algorithmus ein weiteres Mal auf den Bereich "Alles Andere", in dem dann nur noch 3 verschiedene Farben sind, anzuwenden. Letzteres entsprich einfach dem Dutch National Flag Problem.
    
    Dazu muss der Algorithmus aber leicht abgewandelt werden. Er muss:
    
    \begin{itemize}
      \item Die dritte Farbe darüber definieren, dass es weder die erste noch die zweite Farbe ist
      \item Mit verschiedenen Farben, die als Eingabeparameter übergeben werden, arbeiten können
      \item Die Größe des dritten Bereichs zurückliefern
    \end{itemize}
    
    \pagebreak
    
    Hier ist die angepasste Lösung und die beschriebene Verwendung:
    
    \lstinputlisting{Code/4_UCI.cpp}
    
  \end{homeworkProblem}
  
\end{document}