\documentclass{article}

\usepackage{fancyhdr}
\usepackage{extramarks}
\renewcommand{\thispagestyle}[1]{} % damit Titelseite Kopfzeile bekommt
\usepackage{amsmath}
\usepackage{amsthm}
\usepackage{amsfonts}
\usepackage{tikz}
\usepackage[plain]{algorithm}
\usepackage{algpseudocode}
\usepackage{listings}
\lstset{language=C++, numbers=left, frame=single,}
\usepackage[utf8]{inputenc}

\usetikzlibrary{automata,positioning}

%
% Basic Document Settings
%

\topmargin=-0.45in
\evensidemargin=0in
\oddsidemargin=0in
\textwidth=6.5in
\textheight=9.0in
\headsep=0.25in

\linespread{1.1}

\pagestyle{fancy}
\lhead{\hmwkGroup\ -\ \hmwkTitle}
\rhead{\hmwkAuthorName}
\lfoot{\lastxmark}
\cfoot{\thepage}

\renewcommand\headrulewidth{0.4pt}
\renewcommand\footrulewidth{0.4pt}

\setlength\parindent{0pt}

%
% Create Problem Sections
%

\newcommand{\enterProblemHeader}[1]{
    \nobreak\extramarks{}{Aufgabe \arabic{#1} wird auf nächster Seite fortgesetzt\ldots}\nobreak{}
    \nobreak\extramarks{Aufgabe \arabic{#1} (fortgesetzt)}{Aufgabe \arabic{#1} wird auf nächster Seite fortgesetzt\ldots}\nobreak{}
}

\newcommand{\exitProblemHeader}[1]{
    \nobreak\extramarks{Aufgabe \arabic{#1} (fortgesetzt)}{Aufgabe \arabic{#1} wird auf nächster Seite fortgesetzt\ldots}\nobreak{}
    \stepcounter{#1}
    \nobreak\extramarks{Aufgabe \arabic{#1}}{}\nobreak{}
}

\setcounter{secnumdepth}{0}
\newcounter{partCounter}
\newcounter{homeworkProblemCounter}
\setcounter{homeworkProblemCounter}{1}
\nobreak\extramarks{Aufgabe \arabic{homeworkProblemCounter}}{}\nobreak{}

%
% Homework Problem Environment
%
% This environment takes an optional argument. When given, it will adjust the
% problem counter. This is useful for when the problems given for your
% assignment aren't sequential. See the last 3 problems of this template for an
% example.
%
\newenvironment{homeworkProblem}[1][-1]{
    \ifnum#1>0
        \setcounter{homeworkProblemCounter}{#1}
    \fi
    \section{Aufgabe \arabic{homeworkProblemCounter}}
    \setcounter{partCounter}{1}
    \enterProblemHeader{homeworkProblemCounter}
}{
    \exitProblemHeader{homeworkProblemCounter}
}

%
% Homework Details
%   - Title
%   - Due date
%   - Class
%   - Section/Time
%   - Instructor
%   - Author
%

\date{15.04.2018}
\newcommand{\hmwkTitle}{Übungsblatt\ \#1}
\newcommand{\hmwkGroup}{Übungsgruppe 16}
\newcommand{\hmwkDueDate}{19. April 2018}
\newcommand{\hmwkClass}{Datenstrukturen und Algorithmen}
\newcommand{\hmwkClassInstructor}{Joost-Pieter Katoen}
\newcommand{\hmwkAuthorName}{\textbf{Paul Orschau (381085)}}

%
% Title Page
%

\title{
    \vspace{2in}
    \textmd{\textbf{\hmwkClass:\ \hmwkTitle}}\\
    \normalsize\vspace{0.1in}\small{Abgabe\ am\ \hmwkDueDate\ }\\
    \vspace{0.1in}\large{\textit{\hmwkClassInstructor}}
    \vspace{3in}
}

\author{\hmwkAuthorName}

\renewcommand{\part}[1]{\textbf{\large Part \Alph{partCounter}}\stepcounter{partCounter}\\}

%
% Various Helper Commands
%

% Useful for algorithms
\newcommand{\alg}[1]{\textsc{\bfseries \footnotesize #1}}

% For derivatives
\newcommand{\deriv}[1]{\frac{\mathrm{d}}{\mathrm{d}x} (#1)}

% For partial derivatives
\newcommand{\pderiv}[2]{\frac{\partial}{\partial #1} (#2)}

% Integral dx
\newcommand{\dx}{\mathrm{d}x}

% Alias for the Solution section header
\newcommand{\loesung}{\textbf{\large Lösung}}


%
% Start of Document
%

\begin{document}

  \maketitle

  \pagebreak

  \begin{homeworkProblem}[2]
      Zeigen Sie, dass die folgenden Aussagen für beliebige \(n\in\mathbb{N}^{>0}\) gelten:
      
      \begin{enumerate}
          \item \( \sum\limits_{k=1}^n k = \frac{n(n+1)}{2} \)
          \item \( \sum\limits_{i=0}^n 2^i = 2^{n+1}-1 \)
      \end{enumerate}

      \loesung
      
      \textbf{Teil 1}
      
      \( A(n) \Leftrightarrow \sum\limits_{k=1}^n k = \frac{n(n+1)}{2} \)
      
      Beweis von \(A(n)\) per vollständiger Induktion.

      Induktionsanfang: \(n_0=1\)

      \[
          \begin{split}
              A(1) \Leftrightarrow \sum\limits_{k=1}^1 k = 1 = \frac{2}{2} = \frac{1(1+1)}{2}
          \end{split}
      \]

      Induktionsvoraussetzung: Sei \(n\in\mathbb{N}^{>0}\) beliebig, aber fest, und es gelte \(A(n)\).
      
      Induktionsschritt: \(n\rightarrow n+1\)
      
      \[
          \begin{split}
              A(n+1) \Leftrightarrow
              \sum\limits_{k=1}^{n+1} k
              = \sum\limits_{k=1}^n k + (n+1)
              \stackrel{(I.V.)}{=} \frac{n(n+1)}{2} + (n+1)
              = \frac{n(n+1)+2n+2}{2}
              = \frac{n^2+3n+2}{2}
              = \frac{(n+1)(n+2))}{2}
          \end{split}
      \]

      \textbf{Teil 2}
      
      \( A(n) \Leftrightarrow \sum\limits_{i=0}^n 2^i = 2^{n+1}-1 \)
      
      Beweis von \(A(n)\) per vollständiger Induktion.

      Induktionsanfang: \(n_0=1\)

      \[
          \begin{split}
              A(1) \Leftrightarrow \sum\limits_{i=0}^1 2^i
              = 1+2
              = 3
              = 4-1
              = 2^2-1
              = 2^{1+1}-1
          \end{split}
      \]

      Induktionsvoraussetzung: Sei \(n\in\mathbb{N}^{>0}\) beliebig, aber fest, und es gelte \(A(n)\).
      
      Induktionsschritt: \(n\rightarrow n+1\)
      
      \[
          \begin{split}
              A(n+1) \Leftrightarrow
              \sum\limits_{i=0}^{n+1} 2^i
              = \sum\limits_{i=0}^{n} 2^i + 2^{n+1}
              \stackrel{(I.V.)}{=} (2^{n+1}-1) + 2^{n+1}
              = 2*(2^{n+1})-1
              = 2^{n+2}-1
          \end{split}
      \]

  \end{homeworkProblem}

  \pagebreak

  \begin{homeworkProblem}
      Sei \(R\) eine beliebige irreflexive und transitive Relation über einer endlichen Menge \(X\).
      Zeigen oder widerlegen Sie folgende Aussagen immer gelten:
      \begin{enumerate}
        \item \(R\) ist die leere Relation.
        \item \(R\) ist symmetrisch.
        \item \(R\) ist anti-symmetrisch.
      \end{enumerate}
      
      \loesung
      
      $R \text{ irreflexiv} \Leftrightarrow \forall x \in X : \neg(x R x)$
      
      $R \text{ transitiv} \Leftrightarrow \forall x,y,z \in X : (xRy)\wedge(yRz)\implies(xRz)$
      
      Nehme als Beispiel für eine irreflexive und transitive Relation die "Größer als" Relation \(>\)
      und als Beispiel für eine endliche Menge $X=\{1,2\}$.

      \textbf{Teil 1}
      
      Falsch, denn \(>\) erfüllt die Anforderungen an $R$, es handelt sich aber nicht um die leere Relation.
      
      \textbf{Teil 2}
      
      Falsch, denn \(>\) ist nicht symmetrisch, da
      $\exists x=1,y=2 \in X : (x>y)\not\Leftrightarrow(y>x) $
      
      \textbf{Teil 3}
          
      Beweis durch Widerspruch.
      
      Seien $R$ und $X$ nun wieder beliebig.
      
      Widerspruchsannahme: $R$ ist nicht anti-symmetrisch
      
      $\Rightarrow \exists x,y \in X : (xRy)\wedge(yRx)\wedge(x\not=y)
      \stackrel{R\text{ transitiv}}{\Rightarrow} xRx $
      
      Dies widerspricht der Voraussetzung, dass $R$ irreflexiv ist!
      
      Damit ist die Widerspruchsannahme falsch und somit gezeigt, dass $R$ anti-symmetrisch sein muss.
          
  \end{homeworkProblem}

  \pagebreak

  \begin{homeworkProblem}
    
    Gegeben sei eine Liste $L$ der Länge $n>2$, mit natürlichen Zahlen als Eintrag.
    \begin{itemize}
      \item $L$.first beschreibt den ersten Eintrag.
      \item $L$.last beschreibt den letzten Eintrag.
      \item Jeder Eintrag der Liste enthält eine Referenz auf den vorherigen Eintrag, eine Referenz auf den nächsten Eintrag und den Wert des Eintrags.
    \end{itemize}
    Schreiben Sie einen Algorithmus, der die zweitgrößte Zahl in der Liste bestimmt, unter Beachtung der folgenden
    Einschränkung:
    \begin{enumerate}
      \item Der Algorithmus soll iterativ sein (keine Rekursion erlaubt).
      \item Der Algorithmus soll rekursiv sein (keine Schleifen erlaubt).
    \end{enumerate}
    
    \loesung

    Die Lösungen sind in Pseudo-code geschrieben.

    Die Idee besteht darin, die Liste zunächst komplett zu sortieren.

    \textbf{Teil 1}

    \lstinputlisting{Code/4_1_Pseudo.cpp}
    
    Nun enthält secondGreatest die geforderte Referenz auf die zweitgrößte Zahl in der Liste.
    
    \pagebreak
    
    \textbf{Teil 2}

    \lstinputlisting{Code/4_2_Pseudo.cpp}
    
    Nun enthält secondGreatest die geforderte Referenz auf die zweitgrößte Zahl in der Liste.

  \end{homeworkProblem}

\end{document}