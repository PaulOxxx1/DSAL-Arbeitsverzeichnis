\documentclass{article}

\usepackage{fancyhdr}
\usepackage{extramarks}
\renewcommand{\thispagestyle}[1]{} % damit Titelseite Kopfzeile bekommt
\usepackage{amsmath}
\usepackage{amsthm}
\usepackage{amsfonts}
\usepackage{tikz}
\usepackage[plain]{algorithm}
\usepackage{algpseudocode}
\usepackage{listings}
\lstset{language=C++, numbers=left, frame=single,}
\usepackage[utf8]{inputenc}
\usepackage{paralist}
\usepackage{tikz}
\usepackage{multicol}

\usetikzlibrary{automata,positioning}

%
% Basic Document Settings
%

\topmargin=-0.45in
\evensidemargin=0in
\oddsidemargin=0in
\textwidth=6.5in
\textheight=9.0in
\headsep=0.25in

\linespread{1.1}

\pagestyle{fancy}
\lhead{\hmwkGroup\\\hmwkTitle}
\rhead{\hmwkAuthorName}
\lfoot{\lastxmark}
\cfoot{\thepage}

\renewcommand\headrulewidth{0.4pt}
\renewcommand\footrulewidth{0.4pt}

\setlength\parindent{0pt}

%
% Create Problem Sections
%

\newcommand{\enterProblemHeader}[1]{
    \nobreak\extramarks{}{Aufgabe \arabic{#1} wird auf nächster Seite fortgesetzt\ldots}\nobreak{}
    \nobreak\extramarks{Aufgabe \arabic{#1} (fortgesetzt)}{Aufgabe \arabic{#1} wird auf nächster Seite fortgesetzt\ldots}\nobreak{}
}

\newcommand{\exitProblemHeader}[1]{
    \nobreak\extramarks{Aufgabe \arabic{#1} (fortgesetzt)}{Aufgabe \arabic{#1} wird auf nächster Seite fortgesetzt\ldots}\nobreak{}
    \stepcounter{#1}
    \nobreak\extramarks{Aufgabe \arabic{#1}}{}\nobreak{}
}

\setcounter{secnumdepth}{0}
\newcounter{partCounter}
\newcounter{homeworkProblemCounter}
\setcounter{homeworkProblemCounter}{1}
\nobreak\extramarks{Aufgabe \arabic{homeworkProblemCounter}}{}\nobreak{}

%
% Homework Problem Environment
%
% This environment takes an optional argument. When given, it will adjust the
% problem counter. This is useful for when the problems given for your
% assignment aren't sequential. See the last 3 problems of this template for an
% example.
%

\newenvironment{homeworkProblem}[1][-1]{
    \ifnum#1>0
        \setcounter{homeworkProblemCounter}{#1}
    \fi
    \section{Aufgabe \arabic{homeworkProblemCounter}}
    \setcounter{partCounter}{1}
    \enterProblemHeader{homeworkProblemCounter}
}{
    \exitProblemHeader{homeworkProblemCounter}
}

%
% Homework Details
%   - Title
%   - Due date
%   - Class
%   - Section/Time
%   - Instructor
%   - Author
%

\date{30.5.2018}
\newcommand{\hmwkTitle}{Übungsblatt\ \#6}
\newcommand{\hmwkGroup}{Übungsgruppe 16}
\newcommand{\hmwkDueDate}{30. Mai 2018}
\newcommand{\hmwkClass}{Datenstrukturen und Algorithmen}
\newcommand{\hmwkAuthorName}{\textbf{Finn Hess (378104), Jan Knichel (377779), Paul Orschau (381085)}}

%
% Title Page
%

\title{
    \vspace{2in}
    \textmd{\textbf{\hmwkClass:\ \hmwkTitle}}\\
    \normalsize\vspace{0.1in}\small{Abgabe\ am\ \hmwkDueDate\ }\\
    \vspace{3in}
}

\author{\hmwkAuthorName}

\renewcommand{\part}[1]{\textbf{\large Part \Alph{partCounter}}\stepcounter{partCounter}\\}

%
% Various Helper Commands
%

% Useful for algorithms
\newcommand{\alg}[1]{\textsc{\bfseries \footnotesize #1}}

% For derivatives
\newcommand{\deriv}[1]{\frac{\mathrm{d}}{\mathrm{d}x} (#1)}

% For partial derivatives
\newcommand{\pderiv}[2]{\frac{\partial}{\partial #1} (#2)}

% Integral dx
\newcommand{\dx}{\mathrm{d}x}

% Alias for the Solution section header
\newcommand{\loesung}{\textbf{\large Lösung}}

% Parts
\newcommand{\teil}[1]{\vspace{15pt}\textbf{Teil #1}}


%
% Start of Document
%

\begin{document}

  \maketitle

  \pagebreak

  \begin{homeworkProblem}
    
    \begin{itemize}
      \item 9 / 0 / 6 / 2 / 5 / 1 / 4 / 8
      \item 0 / 9 / 6 / 2 / 5 / 1 / 4 / 8
      \item 0 / 9 / 2 / 6 / 5 / 1 / 4 / 8
      \item 0 / 9 / 2 / 6 / 1 / 5 / 4 / 8
      \item 0 / 9 / 2 / 6 / 1 / 5 / 4 / 8
      \item 0 / 2 / 6 / 9 / 1 / 5 / 4 / 8
      \item 0 / 2 / 6 / 9 / 1 / 4 / 5 / 8
      \item 0 / 1 / 2 / 4 / 5 / 6 / 8 / 9
    \end{itemize}
    
  \end{homeworkProblem}

  \begin{homeworkProblem}
    
    \teil{a)}

    \begin{itemize}
      \item 9 / 2 / 9 / 6 / 5 / 1 / 0 / 7
      \item 9 / 2 / 9 / 7 / 5 / 1 / 0 / 6
      \item 9 / 7 / 9 / 2 / 5 / 1 / 0 / 6
      \item 9 / 7 / 9 / 6 / 5 / 1 / 0 / 2
      \item 2 / 7 / 9 / 6 / 5 / 1 / 0 / 9
      \item 9 / 7 / 2 / 6 / 5 / 1 / 0 / 9
      \item 0 / 7 / 2 / 6 / 5 / 1 / 9 / 9
      \item 7 / 0 / 2 / 6 / 5 / 1 / 9 / 9
      \item 7 / 6 / 2 / 0 / 5 / 1 / 9 / 9
      \item 1 / 6 / 2 / 0 / 5 / 7 / 9 / 9
      \item 6 / 1 / 2 / 0 / 5 / 7 / 9 / 9
      \item 6 / 5 / 2 / 0 / 1 / 7 / 9 / 9
      \item 1 / 5 / 2 / 0 / 6 / 7 / 9 / 9
      \item 5 / 1 / 2 / 0 / 6 / 7 / 9 / 9
      \item 0 / 1 / 2 / 5 / 6 / 7 / 9 / 9
      \item 2 / 1 / 0 / 5 / 6 / 7 / 9 / 9
      \item 0 / 1 / 2 / 5 / 6 / 7 / 9 / 9
      \item 1 / 0 / 2 / 5 / 6 / 7 / 9 / 9
      \item 0 / 1 / 2 / 5 / 6 / 7 / 9 / 9
    \end{itemize}
    
  \end{homeworkProblem}

  \begin{homeworkProblem}[3]
    
    \begin{itemize}
      \item 6 / 4 / 4 / 9 / 3 / 2 / 7 / 5
      \item 2 / 4 / 4 / 3 / \textbf{5} / 6 / 7 / 9
      \item 2 / \textbf{3} / 4 / 4 / 5 / 6 / 7 / 9
      \item 2 / 3 / 4 / \textbf{4} / 5 / 6 / 7 / 9
      \item 2 / 3 / 4 / 4 / 5 / 6 / 7 / \textbf{9}
      \item 2 / 3 / 4 / 4 / 5 / 6 / \textbf{7} / 9
    \end{itemize}
    
  \end{homeworkProblem}

  \begin{homeworkProblem}[5]
    
    \teil{a)}
    
    Insertionsort: Da dieser sehr effizient ist für kleine Arrays
    
    \teil{b)}
    
    Countingsort: Da dieser sehr effizient ist für eine geringe Anzahl an Schlüsseln
    
    \teil{c)}
    
    Heapsort: Da dieser $\mathcal{O}$(n*log(n)) für fast sortierte Eingaben hat

    \teil{d)}

    Mergesort: Da dieser Teilprobleme gleichmäßig verteilt und dann zusammenfügt

  \end{homeworkProblem}
  
\end{document}